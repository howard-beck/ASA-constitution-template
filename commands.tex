\usepackage[margin=1in]{geometry}
\setlength{\headheight}{15pt}

\let\oldtitle\title
\renewcommand{\title}[1]{\oldtitle{\textsc{#1}}}

\setlength{\parindent}{0cm}

\usepackage{fancyhdr}
\usepackage{lastpage}

\usepackage{xcolor}

\definecolor{linkblue}{rgb}{0.0, 0.53, 0.74}
\definecolor{urlred}{HTML}{FF1423}
\usepackage{hyperref}
\hypersetup{
    colorlinks = true,
    urlcolor = urlred,
    linkcolor = linkblue
}

%\usepackage{cleveref}
\usepackage[none]{hyphenat}

\fancypagestyle{firststyle}
{
   \fancyhf{}
   \fancyfoot[R]{Page \thepage\ of \pageref*{LastPage}}
   \renewcommand{\headrulewidth}{0pt} % removes horizontal header line
}
\pagestyle{fancy}
\fancyhf{}
\fancyhead[C]{\textsc{\groupname}}
\fancyfoot[R]{Page \thepage\ of \pageref*{LastPage}}



\let\OLDESTSECTION\section

%\setcounter{secnumdepth}{0}
\usepackage{xparse}
\def\nextsecdepth{section}

\makeatletter
\newcounter{labelcount}
\NewDocumentCommand{\nsection}{m m m m m m m O{} O{}}{
    % section counter: #1
    % old section command: #2
    % section pre: #3
    % section ref: #4
    % next step name: #5
    % margin label: #6
    % total label: #7
    % optional: #8
    % header decoration: #9
    
    \ExpandArgs{c}\RenewDocumentCommand{#1}{s m}{%
        \IfBooleanTF{##1}
        {
            #2*{##2}
        }
        {%
            #8%
            \stepcounter{#1}
            \def\secttext{#3. ##2}%
            %
            \def\@currentlabel{\protect{#7}}%
            \phantomsection
            #2*{%
% ignore error on this line, it doesn't exist
                \hyperref[#4]{#9{\textsc{\secttext}}}
                \reversemarginpar\marginpar[%
                    \begin{spacing}{1}%
                        \scriptsize #6%
                    \end{spacing}%
                ]{}%
            }%
            \addcontentsline{toc}{#1}{\secttext}%
            %
            \def\nextsecdepth{#5}%
            \label{#4}%
        }
    }
}
\makeatother

\usepackage{titlesec}
\titlespacing{\section}{0pt}{\parskip}{-\parskip}
\titlespacing{\subsection}{0pt}{\parskip}{-\parskip}
\titlespacing{\subsubsection}{0pt}{\parskip}{-\parskip}

\let\artnum\Roman
\let\secnum\arabic
\let\subsecnum\arabic

\NewCommandCopy{\oldsection}{\section}
\nsection{section}{\oldsection}{Article \artnum{section}}{art.\artnum{section}}{subsection}{ART.\artnum{section}}{Article \artnum{section}}[\vspace{1cm}\hrule][\underline]


\counterwithout{subsection}{section}
\counterwithin*{subsection}{section}
\NewCommandCopy{\oldsubsection}{\subsection}
\nsection{subsection}{\oldsubsection}{Section \secnum{subsection}}{art.\artnum{section}.sec.\secnum{subsection}}{subsubsection}{ART.\artnum{section} \\ SEC.\secnum{subsection}}{Article \artnum{section}, Section \secnum{subsection}}[\vspace{0.5cm}]

\counterwithout{subsubsection}{subsection}
\counterwithin*{subsubsection}{subsection}
\NewCommandCopy{\oldsubsubsection}{\subsubsection}
\nsection{subsubsection}{\oldsubsubsection}{Subsection \subsecnum{subsubsection}}{art.\artnum{section}.sec.\secnum{subsection}.sub.\subsecnum{subsubsection}}{subsubsection}{ART.\artnum{section} \\ SEC.\secnum{subsection} \\ SUB.\subsecnum{subsubsection}}{Article \artnum{section}, Section \secnum{subsection}, Subsection \subsecnum{subsubsection}}
[\vspace{0.25cm}]



\renewcommand{\labelenumii}{\alph{enumii}.}

\newcommand{\officer}[1]{%
    \addcontentsline{toc}{\nextsecdepth}{#1}%
    \hyperref[officer:#1]{\textbf{#1}}%
    \label{officer:#1}%
}


\usepackage{setspace}
%\singlespacing
\linespread{1.25}

\usepackage[shortlabels]{enumitem}
\setlist[itemize]{noitemsep, nolistsep, leftmargin=*}
\setlist[enumerate]{noitemsep, nolistsep, leftmargin=*}

\usepackage{xspace}

\tolerance=1
\emergencystretch=\maxdimen
\hyphenpenalty=10000
\hbadness=10000

\usepackage[titles]{tocloft}
\setlength{\cftbeforesecskip}{0pt}

\let\OLDtableofcontents\tableofcontents
\def\makingTOC{0}

\renewcommand{\tableofcontents}{%
    \def\makingTOC{1}%
    \OLDtableofcontents
}
\renewcommand{\contentsname}{\hfill\bfseries\Large Contents\hfill}
\let\oldmaketitle\maketitle
\renewcommand{\maketitle}{
    \oldmaketitle
    \thispagestyle{firststyle}
    \hrule
    \begin{spacing}{1}
        \tableofcontents
    \end{spacing}
}

\let\oldauthor\author
\renewcommand{\author}[1]{
    \oldauthor{%
        \parbox{\linewidth}{%
            \centering
            \begin{spacing}{1}
                #1
            \end{spacing}
        }
    }
}

\let\olddate\date
\renewcommand{\date}[1]{
    \olddate{%
        \parbox{\linewidth}{%
            \centering
            \begin{spacing}{1}
                #1
            \end{spacing}
        }
    }
}

\let\oldlabel\label
\renewcommand{\label}[1]{
    %\expandafter\let\csname label#1\endcsname\currentlabel
    \oldlabel{#1}
}

\newcommand{\sectref}[1]{%
    %\csname label#1\endcsname
    \labeli
}


\newcommand{\moiralist}[1]{\href{https://groups.mit.edu/webmoira/list/#1}{#1@mit.edu}}
\newcommand{\mailmanlist}[1]{\href{https://mailman.mit.edu:444/mailman/admin/#1}{#1@mit.edu}}
\newcommand{\mailto}[1]{\href{mailto:#1}{#1}}


\newcounter{arrays}
\setcounter{arrays}{1}

\newcommand{\makearr}[1]{%
    \newcounter{arrlen#1}%
    %
    \expandafter\edef\csname#1\endcsname{#1}%
    \stepcounter{arrays}%
}

\newcommand{\getlen}[1]{%
    \expandafter\csname thearrlen#1\endcsname%
}
\newcommand{\setidx}[3]{%
    % array: #1
    % idx: #2
    % value: #3
    %
    \typeout{Setting: #1, at #2: to #3}%
    \def\currentlimit{\the\numexpr\getlen{#1}\relax}%
    \ifnum#2>\currentlimit%
        \PackageError{commands}{Cannot set array index out of bounds}{}%
    \else%
        \expandafter\edef\csname arrval{#1}at{#2}\endcsname{#3}%
        %
        \ifnum#2=\currentlimit%
            \stepcounter{arrlen#1}%
        \fi%
    \fi%
}
\newcommand{\addidx}[2]{%
    \setidx{#1}{\getlen{#1}}{#2}%
}
\newcommand{\getidx}[2]{%
    \expandafter\csname arrval{#1}at{#2}\endcsname%
}
\NewDocumentCommand{\printarr}{m O{0} O{}}{
    \ifnum#2=\getlen{#1}%
        #3%
    \else%
        \edef\starttext{#1 [\getlen{#1}]: }%
        \ifstrempty{#3}{%
        }{%
            \edef\starttext{#3, }%
        }%
        \starttext\printarr{#1}[\the\numexpr1+#2\relax][idx #2: \getidx{#1}{#2}]%
        %\edef\firstidx{\getidx{#1}{0}}
        %\typeout{#1: 0th is \expandafter\@firstofone\expandafter{\the\firstidx}}
        %\starttext idx #2: \getidx{#1}{#2}
    \fi%
}

\makearr{sectionnames}
\makearr{sectionshorts}
\makearr{sectionlongs}
\makearr{sectionnumstyles}

\newcommand{\getdepth}[1]{%
    \expandafter\csname #1depth\endcsname%
}
\newcommand{\getindex}[1]{%
    \expandafter\csname #1index\endcsname%
}

\usepackage{forloop}
\newcounter{copyarrcounter}
\newcommand{\copyarr}[2]{%
    % array to copy: #1
    % name to copy into: #2
    \expandafter\makearr{#2}%
    \forloop{copyarrcounter}{0}{\value{copyarrcounter} < \getlen{#1}}{%
        %\typeout{Attempting to copy #1 into #2}
        \expandafter\addidx\csname#2\endcsname{\getidx{#1}{\thecopyarrcounter}}%
    }%
}

\makearr{SectToTheoDepth}
\newcounter{sectiondepth}
\setcounter{sectiondepth}{-1}

\makearr{SectToSectType}
\makearr{SectToSectNum}

\newcommand{\sectnum}[1]{%
    \expandafter\csname the#1count\endcsname
}


\makeatletter

\newcounter{sectheadnameiter}
\newcounter{dummysectheadcounter}
\NewDocumentCommand{\sectheadname}{m m}{%
    % #1:
    % 1 = full
    % 2 = short
    % 3 = ID
    \edef\returntext{}%
    \edef\ThisDepth{\getidx\SectDepthsAtCount{#2}}%
    \edef\ThisSectToSectType{\expandafter\csname SectToSectTypeAtCount#2\endcsname}%
    \edef\ThisSectToSectNum{\expandafter\csname SectToSectNumAtCount#2\endcsname}%
    %
    \forloop{sectheadnameiter}{0}{\value{sectheadnameiter} < \the\numexpr1+\ThisDepth\relax}{%
        \ifnum\thesectheadnameiter>0%
            \ifnum#1=1%
                \edef\returntext{\returntext, }%
            \else%
                \ifnum#1=2%
                    \edef\returntext{\returntext\\}%
                \else%
                    \edef\returntext{\returntext-}%
                \fi%
            \fi%
        \fi%
        \edef\secttypeID{\getidx\ThisSectToSectType{\thesectheadnameiter}}%
        \edef\indexofsecttype{\getindex{\secttypeID}}%
        \edef\numbr{\getidx\ThisSectToSectNum{\thesectheadnameiter}}%
        \setcounter{dummysectheadcounter}{\numbr}%
        %
        \let\seclist\sectionshorts%
        \def\secspec{.}%
        \ifnum#1=1%
            \let\seclist\sectionlongs%
            \def\secspec{ }%
        \fi%
        %
        \edef\secttypename{\getidx\seclist{\indexofsecttype}}%
        \edef\returntext{{\protect\returntext}\secttypename\secspec\csname\secttypeID numstyle\endcsname {dummysectheadcounter}}%
    }%
    \returntext\xspace
    %\typeout{Testing}
    %\typeout{SectAtCountDepths: \expandafter\@firstofone\expandafter\printedsect}
}
\makeatother
\DeclareRobustCommand{\sectheadfullname}[1]{%
    \sectheadname{1}{#1}%
}
\NewDocumentCommand{\sectheadabbv}{}{%
    \sectheadname{2}{\theheadercounter}%
}
\NewDocumentCommand{\sectheadid}{}{
    \expandafter\sectheadname{3}{0}%
}


\newcounter{lastdepth}
\setcounter{lastdepth}{-1}

\edef\lastTheoDepth{-1}
\usepackage{ifthen}

\newcounter{headercounter}
\setcounter{headercounter}{-1}

\usepackage{etoolbox}

\makearr{SectToSectTypeHistory}
\makearr{SectToSectNumHistory}
\makearr{SectDepthHistory}
\makearr{SectDepthsAtCount}

\makeatletter
\newcommand{\declareHeaderAtCount}[4]{%
    \typeout{Header info [#1]: #2.#3 at #4}
    % headercounter #1
    % SectType #2
    % SectNum #3
    % depth #4
    \setidx\SectToSectTypeHistory{#4}{#2}%
    \setidx\SectToSectNumHistory{#4}{#3}%
    %
    \copyarr\SectToSectTypeHistory{SectToSectTypeAtCount#1}%
    \copyarr\SectToSectNumHistory{SectToSectNumAtCount#1}%
    \setidx\SectDepthsAtCount{#1}{#4}%
}

\usepackage{import}
\import{./}{\jobname.sections}
\newwrite\SECTIONINFO
\immediate\openout\SECTIONINFO=\jobname.sections

\NewDocumentCommand{\makesectioning}{ m m m m O{} }{
    % name: 1
    % short: 2
    % full: 3
    % number type: #4
    % depth: #5
    %
    \edef\nextdepth{\the\numexpr1+\thelastdepth\relax}%
    \ifstrempty{#5}{%
    }{%
        \edef\nextdepth{#5}%
    }%
    \expandafter\edef\csname #1depth\endcsname{\nextdepth}%
    \setcounter{lastdepth}{\nextdepth}%
    %
    \expandafter\edef\csname #1index\endcsname{\getlen{\sectionnames}}%
    %
    \let\commandname\RenewDocumentCommand
    \@ifundefined{#1}{
        \let\commandname\NewDocumentCommand
    }{}
    %\let\commandname\ProvideDocumentCommand
    \ExpandArgs{c}\commandname{#1}{s m}{%
    \ifnum\makingTOC=1
            \OLDESTSECTION*{##2}
            \def\makingTOC{0}
        \else
        
        \edef\TheoDepth{\getdepth{#1}}%
        %
        \ifnum\thesectiondepth>-1
            \ifnum\TheoDepth>\the\numexpr1+\lastTheoDepth\relax
                \PackageError{commands}{Cannot use #1, please use section headers sequentially}{}%
            \fi%
        \fi
        %
        \ifnum\lastTheoDepth<\TheoDepth%
            \stepcounter{sectiondepth}%
        \else%
            \addtocounter{sectiondepth}{\the\numexpr \TheoDepth-\lastTheoDepth\relax}%
        \fi%
        %
        \edef\lastTheoDepth{\TheoDepth}
        \setidx\SectToTheoDepth{\thesectiondepth}{\TheoDepth}%
        %
        \stepcounter{#1count}%
        \setidx\SectToSectType{\thesectiondepth}{#1}%
        \setidx\SectToSectNum{\thesectiondepth}{\sectnum{#1}}%
        %
        \stepcounter{headercounter}%
        
        %\copyarr{\SectToSectType}{SectToSectTypeAtCount\theheadercounter}
        %\copyarr{\SectToSectNum}{SectToSectNumAtCount\theheadercounter}
        %\addidx\SectAtCountDepths{\thesectiondepth}
        \protected@write\SECTIONINFO{}{\string%
            \declareHeaderAtCount{\theheadercounter}{#1}{\sectnum{#1}}{\thesectiondepth}%
        }
        %\expandafter\edef\csname SectDepthAtCount\theheadercounter\endcsname{\thesectiondepth}
        %
        \def\spacingamount{\fpeval{1/(2^\TheoDepth)}}%
        \ifnum\thesectiondepth=0%
            \vspace{1cm}
            \hrule
        \fi%
        \par
        \vspace{\spacingamount cm}%
        %\ifnum\thesectiondepth=0
        %    \hrule%
        %\fi%
        %
        %
        %%% TEXT HERE
        \phantomsection%
        \def\sectionstyling{DEFAULTSECTIONSTYLE}%
        \ifnum\thesectiondepth<\getlen{\SectionStyling}%
            \def\sectionstyling{\getidx\SectionStyling{\thesectiondepth}}%
        \fi%
        \hyperref[secthead\theheadercounter]{\csname \sectionstyling\endcsname{\textsc{#3 #4{#1count}. ##2}}}%
        \reversemarginpar\marginpar[%
            \begin{spacing}{1}%
                \scriptsize \sectheadabbv%
            \end{spacing}%
        ]{}
        \iffalse
        \protected@write \@auxout {}{%
            \string \newlabel{secthead\theheadercounter}{%
                {\expandafter\sectheadfullname}%
                {\thepage}%
                {#1}%
                {\sectheadid}%
                {}%
            }%
        }%
        \fi
        \edef\@currentlabel{\sectheadfullname{\theheadercounter}}
        \label{secthead\theheadercounter}%
        \par
        %\\%
        %
        \IfBooleanTF{##1}
        {}
        {
            \ifnum\thesectiondepth<\getlen{\ToCDepths}%
                \addcontentsline{toc}{\getidx\ToCDepths{\thesectiondepth}}{#3 #4{#1count}. ##2}%
            \fi%
        }



        \fi
    }%
    %
    \def\numsectypes{\getlen{\sectionnames}}%
    \newcounter{#1count}%
    \ifnum\numsectypes>0%
        \edef\lastsecnum{\the\numexpr\numsectypes - 1\relax}%
        \counterwithin*{#1count}{\getidx\sectionnames{\lastsecnum}count}%
    \fi%
    %
    \addidx\sectionnames{#1}%
    \addidx\sectionshorts{#2}%
    \addidx\sectionlongs{#3}%
    \expandafter\let\csname#1numstyle\endcsname#4
}
\makeatother

\makearr{ToCDepths}
\makearr{SectionStyling}

\addidx\ToCDepths{section}
\def\SECTIONSTYLE{\normalfont\Large\bfseries\underline}
\addidx\SectionStyling{SECTIONSTYLE}

\addidx\ToCDepths{subsection}
\def\SUBSECTIONSTYLE{\normalfont\large\bfseries}
\addidx\SectionStyling{SUBSECTIONSTYLE}

\addidx\ToCDepths{subsubsection}
\def\DEFAULTSECTIONSTYLE{\normalfont\normalsize\bfseries}

\addidx\ToCDepths{paragraph}
%\addidx\SectionStyling{\normalfont\normalsize\bfseries}

\addidx\ToCDepths{subparagraph}
%\addidx\SectionStyling{\normalfont\normalsize\bfseries}



\let\appnum\Alph
\makesectioning{appendix}{APP}{Appendix}{\appnum}

\let\artnum\Roman
\makesectioning{article}{ART}{Article}{\artnum}[\thelastdepth]

\let\secnum\arabic
\makesectioning{section}{SEC}{Section}{\secnum}

\let\subnum\arabic
\makesectioning{subsection}{SUB}{Subsection}{\subnum}