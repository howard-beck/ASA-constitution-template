% !TeX program = xelatex
\documentclass{constitution}


%%% NOTE: the following code puts the document in
%   Times New Roman. If you want to keep it, please
%   switch your compiler to XeTeX.
%   If you are on Overleaf, see:
%   https://www.overleaf.com/learn/how-to/Changing_compiler

\usepackage{fontspec}
% Gyre Termes is TeX's copy of Times New Roman
% We don't use Times New Roman (also available)
% as it doesn't support script fonts
\setmainfont{TeX Gyre Termes}

%%% NOTE: Here you can change the section numbering
%   scheme. The following are the defaults which
%   you can edit by uncommenting. see:
%   https://www.overleaf.com/learn/latex/Counters#Accessing_and_printing_counter_values

%\let\artnum\Roman
%\let\secnum\arabic
%\let\subsecnum\arabic

\def\groupname{[Group Name]}
\begin{document}
\title{\groupname Constitution}
%%% OPTIONAL:
\author{%
    Membership list: \moiralist{your-club-list} \\
    Exec list: \mailto{your-club-exec@mit.edu} \\
    Website: \url{https://your-club.mit.edu/}
}
\date{\textit{Last updated}: \today}

\maketitle

\tableofcontents


%%% REQUIREMENTS: https://asa.mit.edu/start-group/constitution-requirements


\article{Purpose}
%%% REQUIRED: The purpose of your organization
The purpose of the \groupname is...

\article{Membership}
\begin{enumerate}
    %%% REQUIRED: Any member of the MIT student community must be eligible for membership
    \item Any member of the MIT community is eligible to become a member of this organization.
    One is considered a member of this organization in any given academic term if they have attended at least one club activity and one meeting in the preceding or current term.

    %%% REQUIRED: The organization's membership will at all times consist of at least 5 MIT students and be more than half MIT students.
    \item The organization's membership will at all times consist of at least 5 MIT students and be more than half MIT students.

    %%% optional
    \item If this organization charges any monetary dues, there will be exceptions made for students that cannot afford these dues.
\end{enumerate}

\section{MIT Nondiscrimination Policy}
%%% REQUIRED
The organization shall not discriminate based on any characteristic listed in the \href{https://handbook.mit.edu/nondiscrimination}{MIT~Nondiscrimination~Policy} for membership, officer position, or in any other aspect.

\section{Membership Removal}
%%% REQUIRED: A clause for membership removal (new requirement)
A member can be removed by a two-third vote by all the other members.
The member to be removed will be given the opportunity to speak at the meeting at which the vote takes place.

\article{Officers}
\begin{enumerate}
    %%% REQUIRED: Must include at least president and treasurer (or corresponding positions) and those two positions must be required to be distinct current MIT students
    \item This organization will have at a minimum two officers. The president and the treasurer will be distinct current MIT students.
    \begin{enumerate}
        %%% REQUIRED: a president
        \item \officer{President}
        \begin{enumerate}
            \item The president shall be a currently enrolled MIT student.
            
            \item The president shall be the official representative of the group to any other organization and to MIT.
            
            \item The president will preside over all meetings.
            
            \item The president is responsible for determining when meetings are and publicizing this to the group.
        \end{enumerate}

        %%% REQUIRED: a treasurer
        \item \officer{Treasurer}
        \begin{enumerate}
            \item The treasurer shall be a currently enrolled MIT student.
            
            \item The treasurer shall be responsible for the finances of the group.
            
            \item The treasurer must approve all disbursements from the group's accounts, including reimbursements to members.
        \end{enumerate}
    \end{enumerate}

    %%% REQUIRED: The executive board must be comprised of at least half current MIT students
    \item At least half of the individuals who are officers of this organization must be current MIT students.
\end{enumerate}

\section{Elections}
%%% REQUIRED: Officer election procedure and dates (including what makes a quorum)
\begin{enumerate}
    %%% Required: Officer election procedure and dates (including what makes a quorum)
    \item Any member of this organization is eligible to run for any office.
    \item Quorum for elections is two-thirds of the group.
    \item Any member is elected if they win a majority of the voting members.
    \item If more than two people are running and no one wins a majority, then the person with the fewest votes is dropped from the ballot and votes are recast.
    
    %%% REQUIRED: Explicit terms of office of your group's officers that encompass every day of the year
    \item Elections of officers shall occur [when].

    %%% REQUIRED: Explicit terms of office of your group's officers that encompass every day of the year
    \item The term of office runs from one election to the subsequent election.
\end{enumerate}

\section{Removal}
%%% REQUIRED: Removal process for officers
Officers may be removed by a two-thirds vote of the members.

\article{Meetings}
\begin{enumerate}
    %%% REQUIRED: The frequency of meetings
    \item Meetings shall be held at least every [time period].

    %%% REQUIRED: Who runs the meetings
    \item Meetings shall be presided over by the president, unless they are absent, and in that case the treasurer shall preside.

    %%% REQUIRED: How decisions are made during a meeting (including what makes a quorum)
    \item All decisions shall be made by a majority vote of all members present.
    \item Quorum for a meeting shall be one-quarter of the members of the organization.
\end{enumerate}

\article{Amendments}
\begin{enumerate}
    %%% REQUIRED: How the amendment is presented
    \item Amendments shall be presented by any member of the organization.

    %%% REQUIRED: How the amendment is passed
    \item Amendments shall be passed by a two-thirds voted of the members present.

    %%% REQUIRED: The quorum necessary for an amendment to be passed
    \item Quorum for amending this constitution shall be two-thirds of all members of the organization.
\end{enumerate}

\article{ASA Governance Clause}
%%% REQUIRED: The [your group name] agrees to abide by the rules and regulations of the Association of Student Activities, and its executive board. This constitution, amendments to it, and the by-laws of this organization shall be subject to review by the ASA Executive Board to ensure that they are in accordance with the aforementioned rules and regulations.
The \groupname agrees to abide by the \href{https://asa.mit.edu/asa-policies}{rules~and~regulations} of the \href{https://asa.mit.edu/}{Association~of~Student~Activities}, and its \href{https://asa.mit.edu/about-asa/board-members}{executive~board}.
This constitution, amendments to it, and the by-laws of this organization shall be subject to review by the ASA Executive Board to ensure that they are in accordance with the aforementioned rules and regulations.



\appendix{TeX help -- Remove me for final version}
\article{Compiling issues}
Please note: you may need to compile the document up to $ 4 $ times after adding a new
section (or article, subsection, etc).
This is in order for the document to update its internal catalog of all the section references.
\\\\
Additionally, if you do not define \verb|\groupname|, the program will throw an error.
\verb|\groupname| is a macro used to store your group name, to put in the title, in the PDF metadata, and in the page headings.
To define it, use\par
\quad \verb|\def\groupname{Association of Student Activities}|\par
where you should instead put your group name.

\article{Useful commands}
The following commands are defined for your convenience.
\\\\
To type a WebMoira mailing list, you can use \verb|\moiralist|.
For example:\par
    \quad \verb|\moiralist{asa-exec}|\par
displays as \moiralist{asa-exec}, which also links to the WebMoira site to join the list.
The same goes for: \par
    \quad \verb|\mailmanlist{asa-exec}|\par
which displays as \mailmanlist{asa-exec} and links to the mailman site (although note that the corresponding mailman website does not exist, in this case).
\\\\
If you want to link to a mailing list without letting the reader add themselves to the list, you can use\par
\quad \verb|\mailto{asa-exec@mit.edu}|\par
which displays as \mailto{asa-exec@mit.edu} and links to the \verb|mailto:| site correctly.
That is, by clicking on the link you are redirected to an email provider to send an email to the list.
\\\\
Additionally, you may wish to put officer definitions in the table of contents (and navigation sidebar) while they may not deserve their own (sub...)section.
For this purpose, there is a \verb|\officer| command.
For example,\par
\quad \verb|The \officer{Vice President} is important|\par
displays as\par
\quad The \officer{Vice President} is important.\par
You can check the table of contents or navigation bar to see that it says [Officer] President.
It is put at the same depth as the previous section header.
\\\\
While this constitution template is mainly to be used for constitutions, you may use it for any sort of club documentation you see fit.
For this purpose, there is a \verb|\doctype| command.
It is used in the title, and in the PDF's metadata.
To change it, write, somewhere in the preamble,\par
\quad \verb|\def\doctype{Your Document Type Here}| (default is Constitution)

\article{Section headers}
\label{headers}

\section{Section header list}
The constitution template naturally supports the following section header types:
\begin{enumerate}
    \item \verb|\appendix|
    \item \verb|\article|
    \item \verb|\section|
    \item \verb|\subsection|
    \item \verb|\subsubsection|
\end{enumerate}
They are indented in precisely that order, and are used exactly like the standard TeX \verb|\section| command.
If you for some reason need to go deeper than that, you should use \verb|\makeNsubsection{3}| or some other number -- in the preamble.
This will define, for example, the command \verb|\subsubsubsection|.

\subsection{Custom section headers}
For a custom section header command, please use\par
\quad \verb|\makesectioning{sectioncommand}{ABBV}{Sectionname}|\par
For example, \verb|\section| is defined using\par
\quad \verb|\makesectioning{section}{SEC}{Section}|

\section{Styling}
\subsection{Section header font}
You can modify the font sizes and styles.
In the template, they are defined as follows:\par
\quad \verb|\def\appendixstyle{\normalfont\Large\bfseries\scshape}|\par
\quad \verb|\def\articlestyle{\normalfont\Large\bfseries\scshape\underline}|\par
\quad \verb|\def\sectionstyle{\normalfont\large\bfseries\scshape}|\par
\quad \verb|\def\defaultstyle{\normalfont\normalsize\bfseries\scshape}|\par
Any style not defined will use the \verb|defaultstyle|.
\\\\
You can use similar commands at any point throughout the document.
To apply your changes to the entire document, put the customizations in the preamble.
For example, at this point, we can change what the next subsubsection will look like.
This code will make it be {\Huge Huge} font in \textsc{SmallCaps}:\par
\quad \verb|\def\subsubsectionstyle{\normalfont\Huge\scshape}|\par
\def\subsubsectionstyle{\normalfont\Huge\scshape}
\subsubsection{As is demonstrated here}
\let\subsubsectionstyle\defaultstyle
You can learn more about font families and styling \href{https://www.overleaf.com/learn/latex/Font_sizes%2C_families%2C_and_styles}{here}.

\subsection{Section numbering}
The default numbering system is given by:\par
\quad \verb|\let\appendixnumstyle\Alph|\par
\quad \verb|\let\articlenumstyle\Roman|\par
\quad \verb|\let\sectionnumstyle\arabic|\par
\quad \verb|\let\defaultnumstyle\arabic|\par
That is -- appendices are indexed by capital alphabetic letters (A, B, ...), articles by capital roman numerals (i, ii, ...), sections by arabic numerals (1, 2, ...).
If a style is not specified, the default of arabic is used.
\\\\
As an example, a \verb|\subsubsectionnumstyle| is not specified, so that the next subsubsection will use arabic numerals
\subsubsection{like so}
However, we can use Roman for subsubsections if we so choose:
\quad \verb|\let\subsubsectionnumstyle\Roman|\par
\let\subsubsectionnumstyle\Roman
\subsubsection{resulting in this}
\let\subsubsectionnumstyle\defaultnumstyle

\subsection{Line breaks and suffix}
Some clubs may prefer to keep their section headers inline.
In this case, they may set, in the preamble\par
\quad \verb|\setbool{sectionlinebreak}{false}|\par
\setbool{sectionlinebreak}{false}
\subsubsection{which will look like this}
Where future text stays inline.
However, the separation between the section header and the text can look awkward, so you can put a suffix using\par
\quad \verb|\def\afterheadertext{.}|
\def\afterheadertext{.}
\subsubsection{resulting in this}
Where future text is now separated by a ``\verb|.|", for example.
\setbool{sectionlinebreak}{true}
\def\afterheadertext{}

\def\maxheaderfullinfosize{3}
\section{Section header labels}
To keep track of which article/etc you are in, there are, by default, labels on the left margin.
To instead put them to the right of the section header, you can declare:\par
\quad \verb|\def\headerfullinfoside{inline}|\par
which will
\def\headerfullinfoside{inline}
\subsection{look like this}
Or, to disable it altogether, you can write\par
\quad \verb|\def\headerfullinfoside{none}|\par
\def\headerfullinfoside{none}
\subsubsection{to get this}
The default value is \verb|margin|.
When it is in the margin, you may limit the amount of data, so that the margin notes do not collide.
The default maximum depth is \verb|3|, but you can expand it.
For example, going up to \verb|10| will likely guarantee you capture all information:\par
\quad \verb|\def\maxheaderfullinfosize{10}|
\def\maxheaderfullinfosize{10}
\def\headerfullinfoside{margin}
\subsubsection{like so}
\def\maxheaderfullinfosize{3}
\vspace{1.5cm}

\subsection{Appendix Number Collisions}
See \ref{number:col}

\subsection{Further styling}
You may have more stylistic preferences not captured here.
There are commands that are run before and after every section header, which you can redefine as you wish.
For example, to put a horizontal line with some extra spacing above and below every subsubsection, you can use\par
\quad \verb|\def\subsubsectionpre{\hrule\vspace{1cm}}|\par
\quad \verb|\def\subsubsectionpost{\vspace{1cm}\hrule}|\par
\def\subsubsectionpre{\hrule\vspace{1cm}}
\def\subsubsectionpost{\vspace{1cm}\hrule}
\subsubsection{which looks like this}
\def\subsubsectionpre{}
\def\subsubsectionpost{}

\article{Referencing}
\label{number:col}
To refer to another section, you should use the standard \verb|\label|-\verb|\ref| system from LaTeX.
The labels have been redefined so as to not just output a number, but the section header name.
For example, to learn about typing section headers, you should go to \ref{headers}.
This was generated with the code:\par
\quad \verb|\article{Section headers}|\par
\quad \verb|\label{headers}|\par
\quad \verb|...|\par
\quad \verb|For example, to learn about typing section headers, you should go to \ref{headers}.|\par

\section{Compact References}
By default, the references are \textit{compact}, meaning that, for example, text in Article II, Section 3 referencing Article II, Section 5, will only display ``Section 5" in a reference to Article II, Section 5.
If you want to fully expand out references, you can set:\par
\quad \verb|\setbool{MakeCompactRefs}{false}|\par
\setbool{MakeCompactRefs}{false}
For example, using this, you would learn about section headings at \ref{headers}
instead of at just
\setbool{MakeCompactRefs}{true}%
\ref{headers}.
\\\\
The issue with compact references is that referencing Article II from Appendix A, Article III, for example, is ambiguous.
It could be [Main Text] Article II or Appendix A, Article II.
For this reason, article numbers in appendices are prefixed with the appendix number by default.
This can be disabled as follows:\par
\quad \verb|\setbool{ArticleWithAppendixPrepend}{false}|\par
\setbool{ArticleWithAppendixPrepend}{false}
Then, a (full) reference would look like
\setbool{MakeCompactRefs}{false}%
\ref{headers} as opposed to
\setbool{ArticleWithAppendixPrepend}{true}%
\ref{headers}.
\setbool{MakeCompactRefs}{true}

\end{document}