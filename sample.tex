% !TeX program = xelatex
\documentclass{constitution}


%%% NOTE: the following code puts the document in
%   Times New Roman. If you want to keep it, please
%   switch your compiler to XeTeX.
%   If you are on Overleaf, see:
%   https://www.overleaf.com/learn/how-to/Changing_compiler

\usepackage{fontspec}
% Gyre Termes is TeX's copy of Times New Roman
% We don't use Times New Roman (also available)
% as it doesn't support script fonts
\setmainfont{TeX Gyre Termes}

%%% NOTE: Here you can change the section numbering
%   scheme. The following are the defaults which
%   you can edit by uncommenting. see:
%   https://www.overleaf.com/learn/latex/Counters#Accessing_and_printing_counter_values

%\let\artnum\Roman
%\let\secnum\arabic
%\let\subsecnum\arabic

\def\groupname{[Group Name]}
\begin{document}
\title{\groupname Constitution}
%%% OPTIONAL:
\author{%
    Membership list: \moiralist{your-club-list} \\
    Exec list: \mailto{your-club-exec@mit.edu} \\
    Website: \url{https://your-club.mit.edu/}
}
\date{\textit{Last updated}: \today}

\maketitle

\tableofcontents


%%% REQUIREMENTS: https://asa.mit.edu/start-group/constitution-requirements


\article{Purpose}
%%% REQUIRED: The purpose of your organization
The purpose of the \groupname is...

\article{Membership}
\begin{enumerate}
    %%% REQUIRED: Any member of the MIT student community must be eligible for membership
    \item Any member of the MIT community is eligible to become a member of this organization.
    One is considered a member of this organization in any given academic term if they have attended at least one club activity and one meeting in the preceding or current term.

    %%% REQUIRED: The organization's membership will at all times consist of at least 5 MIT students and be more than half MIT students.
    \item The organization's membership will at all times consist of at least 5 MIT students and be more than half MIT students.

    %%% optional
    \item If this organization charges any monetary dues, there will be exceptions made for students that cannot afford these dues.
\end{enumerate}

\section{MIT Nondiscrimination Policy}
%%% REQUIRED
The organization shall not discriminate based on any characteristic listed in the \href{https://handbook.mit.edu/nondiscrimination}{MIT~Nondiscrimination~Policy} for membership, officer position, or in any other aspect.

\section{Membership Removal}
%%% REQUIRED: A clause for membership removal (new requirement)
A member can be removed by a two-third vote by all the other members.
The member to be removed will be given the opportunity to speak at the meeting at which the vote takes place.

\article{Officers}
\begin{enumerate}
    %%% REQUIRED: Must include at least president and treasurer (or corresponding positions) and those two positions must be required to be distinct current MIT students
    \item This organization will have at a minimum two officers. The president and the treasurer will be distinct current MIT students.
    \begin{enumerate}
        %%% REQUIRED: a president
        \item \officer{President}
        \begin{enumerate}
            \item The president shall be a currently enrolled MIT student.
            
            \item The president shall be the official representative of the group to any other organization and to MIT.
            
            \item The president will preside over all meetings.
            
            \item The president is responsible for determining when meetings are and publicizing this to the group.
        \end{enumerate}

        %%% REQUIRED: a treasurer
        \item \officer{Treasurer}
        \begin{enumerate}
            \item The treasurer shall be a currently enrolled MIT student.
            
            \item The treasurer shall be responsible for the finances of the group.
            
            \item The treasurer must approve all disbursements from the group's accounts, including reimbursements to members.
        \end{enumerate}
    \end{enumerate}

    %%% REQUIRED: The executive board must be comprised of at least half current MIT students
    \item At least half of the individuals who are officers of this organization must be current MIT students.
\end{enumerate}

\section{Elections}
%%% REQUIRED: Officer election procedure and dates (including what makes a quorum)
\begin{enumerate}
    %%% Required: Officer election procedure and dates (including what makes a quorum)
    \item Any member of this organization is eligible to run for any office.
    \item Quorum for elections is two-thirds of the group.
    \item Any member is elected if they win a majority of the voting members.
    \item If more than two people are running and no one wins a majority, then the person with the fewest votes is dropped from the ballot and votes are recast.
    
    %%% REQUIRED: Explicit terms of office of your group's officers that encompass every day of the year
    \item Elections of officers shall occur [when].

    %%% REQUIRED: Explicit terms of office of your group's officers that encompass every day of the year
    \item The term of office runs from one election to the subsequent election.
\end{enumerate}

\section{Removal}
%%% REQUIRED: Removal process for officers
Officers may be removed by a two-thirds vote of the members.

\article{Meetings}
\begin{enumerate}
    %%% REQUIRED: The frequency of meetings
    \item Meetings shall be held at least every [time period].

    %%% REQUIRED: Who runs the meetings
    \item Meetings shall be presided over by the president, unless they are absent, and in that case the treasurer shall preside.

    %%% REQUIRED: How decisions are made during a meeting (including what makes a quorum)
    \item All decisions shall be made by a majority vote of all members present.
    \item Quorum for a meeting shall be one-quarter of the members of the organization.
\end{enumerate}

\article{Amendments}
\begin{enumerate}
    %%% REQUIRED: How the amendment is presented
    \item Amendments shall be presented by any member of the organization.

    %%% REQUIRED: How the amendment is passed
    \item Amendments shall be passed by a two-thirds voted of the members present.

    %%% REQUIRED: The quorum necessary for an amendment to be passed
    \item Quorum for amending this constitution shall be two-thirds of all members of the organization.
\end{enumerate}

\article{ASA Governance Clause}
%%% REQUIRED: The [your group name] agrees to abide by the rules and regulations of the Association of Student Activities, and its executive board. This constitution, amendments to it, and the by-laws of this organization shall be subject to review by the ASA Executive Board to ensure that they are in accordance with the aforementioned rules and regulations.
The \groupname agrees to abide by the \href{https://asa.mit.edu/asa-policies}{rules~and~regulations} of the \href{https://asa.mit.edu/}{Association~of~Student~Activities}, and its \href{https://asa.mit.edu/about-asa/board-members}{executive~board}.
This constitution, amendments to it, and the by-laws of this organization shall be subject to review by the ASA Executive Board to ensure that they are in accordance with the aforementioned rules and regulations.

%%% OPTIONAL, feel free to remove
\article{Resources}
\section{ASA Operating Guidelines}
\url{https://asa.mit.edu/asa-policies/asa-guidelines-and-bylaws}

\section{Massachusetts Hazing Law}
\href{https://malegislature.gov/Laws/GeneralLaws/PartIV/TitleI/Chapter269}{269}:\href{https://malegislature.gov/Laws/GeneralLaws/PartIV/TitleI/Chapter269/Section17}{17},\href{https://malegislature.gov/Laws/GeneralLaws/PartIV/TitleI/Chapter269/Section18}{18},\href{https://malegislature.gov/Laws/GeneralLaws/PartIV/TitleI/Chapter269/Section19}{19}

\section{SOLE Student Organization Handbook}
\url{https://studentlife.mit.edu/system/files/2024-02/20240220-sole-handbook_2.pdf}

\article{TeX help -- remove me}
\section{Sectioning commands}
\verb|\section| has been repurposed to mean ``Article."
\\
\\
(Hopefully not too confusingly), \verb|\subsection| has been repurposed to mean ``Section," and \verb|\subsubsection| means ``Subsection."
\\
\\
All sectioning commands have been modified to be clickable.
See \ref{justification}

\subsection{Justification}
\label{justification}
The motivation for this change is to make it more intuitive to type the document, as you would use sectioning commands as usual.
Another reason is that the Overleaf interface displays a file outline based on \verb|\section| and its relatives, so electing to createan \verb|\article| command and not touching \verb|\section| and \verb|\subsection| means the outline would not display.

\section{File System}
\verb|commands.tex| contains all the custom and redefined commands, along with almost all the package imports -- so as to not clutter this file.
\\\\
\verb|latexmkrc| contains information that tells the compiler to use Eastern time in determining the date.
\\\\
\verb|main.tex| is what you think it is.

\section{Referencing}
You can reference other sections with \verb|\ref| as usual.
It has been redefined to automatically display the full name of the section.
\\\\
For example, take a look at \ref{justification}.



%\let\section\mysection

\article{Testing}
hi
\section{Besting}
a
\section{Cesting}

\article{Desting}
\section{Eesting}

\article{Festing}
\section{Gesting}
\subsection{Hesting}

\appendix{Iesting}
\article{Pesting}
\section{Westing}

\section{Testing}

\subsection{Resting}

\end{document}